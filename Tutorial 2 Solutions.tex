\documentclass{article}
\usepackage{amsmath, amssymb, amsfonts, amsthm, mathtools}
\usepackage[utf8]{inputenc}
\usepackage[inline]{enumitem}
\usepackage{cancel}
\usepackage{soul}
\usepackage{hyperref}
\newtheorem{theorem}{Theorem}
\newtheorem{lem}{Lemma}
\newtheorem{defn}{Definition}

\setlength\parindent{0pt}

\usepackage{geometry}
\geometry{
	a4paper,
	total={170mm,257mm},
	left=20mm,
	top=20mm,
}

\title{CS 228: Tutorial Solutions}
\author{Aryaman Maithani}
\date{}

\begin{document}
\maketitle
\begin{center}
\textbf{Problem Set 2}
\end{center}
1. $\mathcal{P}$ is not consistent. \\
\emph{Proof.} For any $p,$ the following are well-formed formulae: $p \vee \neg p$ and $\neg(p \vee\neg p).$ Let $A$ be the former and $B$ be the latter. By hypothesis, $A \to B$ is an axiom. We now derive $\perp$ using this.
\begin{center}
	\begin{tabular}{l l l}
	1. & $p \vee \neg p \to \neg(p \vee \neg p)$ & Axiom\\
	2. & $p \vee \neg p$ & LEM\\
	3. & $\neg(p \vee \neg p)$ & MP 1, 2\\
	4. & $\perp$ & $\perp$ i 2, 3
	\end{tabular}
\end{center}
Thus, by definition, $\mathcal{P}$ is inconsistent.

\hrulefill

2. (a)\\
(i) To show that $\mathcal{A} = \{\neg, \wedge\}$ is an adequate set of formulae. \\
Note that $A\vee B \equiv \neg(\neg A \wedge \neg B).$ Thus, we can construct $\vee$ using $\mathcal{A}.$ \\
Also, $A \to B \equiv \neg A \vee B.$ \\
As we had already constructed $\vee$ using $\mathcal{A},$ we are done.\\~\\
(ii) To show that $\mathcal{B} = \{\neg, \to\}$ is an adequate set of formulae.\\
Note that $A\vee B \equiv \neg A \to B.$ Thus, we can construct $\vee$ using $\mathcal{B}.$ \\
Also, $A \wedge B\equiv \neg(\neg A\vee \neg B).$ \\
As we had already constructed $\vee$ using $\mathcal{B},$ we are done.\\~\\
(iii) To show that $\mathcal{C} = \{\to, \perp\}$ is an adequate set of formulae.\\
Note that $\neg A \equiv A \to \perp.$ Thus, we can construct $\neg$ using $\mathcal{C}.$\\
The result now follows from what we proved earlier as we can construct everything else using $\to$ and $\neg.$ \\~\\
%
2. (b)\\
We shall prove the statement by way of contradiction. Suppose not. That is, $C$ is adequate and $\neg \not\in C$ and $\perp\not\in C.$\\
Let us look at the set of all formulae $S$ that contain only $a$ as a propositional atom and connectives from $C.$ By hypothesis, $C$ is adequate, that is, for every formula, there is an equivalent formula with only connectives from $C.$ In particular, an equivalent of the formula $\neg a$ should be in $S.$ Note that $\neg a$ evaluates to $F$ whenever $a$ is $T.$\\~\\
\textbf{Claim:} Given any formula $\varphi \in S,$ it must evaluate to $T$ whenever $a$ is $T.$ This will show that $C$ is not adequate (as it could not possibly have $\neg a$) and complete our proof.\\~\\
\emph{Proof.} We shall prove this by inducting on the number $n$ of connectives in $\varphi.$\\
Base case. $n = 0.$ The only such formula is $\varphi = a.$ For this, the claim holds trivially.\\
Inductive hypothesis. Given a formula $\varphi$ such that it has $n$ or fewer connectives, $\varphi$ evaluates to $T$ when $a$ is $T.$\\
Inductive step. Suppose $\varphi$ is a formula with $n+1$ connectives. Then, $\varphi$ can be written as $\varphi_1 \circ \varphi_2$ where $\circ$ is some connective from $C$ and $\varphi_1,\;\varphi_2$ are formulae in $S$ with at most $n$ connectives. By inductive hypothesis, $\varphi_1$ and $\varphi_2$ both evaluate to $T$ whenever $a$ is $T.$ Let us now consider the following partial truth table.\\
\begin{center}

\begin{tabular}{|c|c|c|c|c|c|c|}
	\hline
	$a$ & $\varphi_1$ & $\varphi_2$ & $\varphi_1 \wedge \varphi_2$ & $\varphi_1 \vee \varphi_2$ & $\varphi_1 \to \varphi_2$ & $\varphi_2 \to \varphi_1$ \\
	\hline
	$T$ & $T$ & $T$ & $T$ & $T$ & $T$ & $T$ \\
	\hline
\end{tabular}

\end{center}

Thus, it is clear that no matter what $\circ$ is, $\varphi$ will always evaluate to $T$ whenever $a$ is $T.$\\
This completes our proof of the claim, which in turns completes the complete proof.

\hrulefill
\newpage

3. We show that $\{\downarrow\}$ is adequate by observing the following:
\begin{enumerate}[nosep] 
	\item $\neg A \equiv A\downarrow A$
	\item $A \wedge B \equiv \neg(A \downarrow B)$
	\item $A \vee B \equiv \neg(\neg A \wedge \neg B)$
\end{enumerate}
Hence, proved.

\hrulefill

4. We show that $\{\oplus\}$ is not adequate by way of contradiction. Assume that $\{\oplus\}$ were adequate. Then, we should have been able to write $A \wedge B$ in terms of $A, B$ and $\oplus.$ We show that this is not possible using the following claim.\\~\\
\textbf{Claim:} Let $\varphi$ be any formula made using $A, B$ and $\oplus.$ Then, the truth table of $\varphi$ has an even number of $T$s.\\
\emph{Proof.} We prove the claim using induction. We shall induct on the number $n$ of $\oplus$s in the formula.\\
Base case. $n = 0.$ The only such formulae are $A$ and $B.$
\begin{center}

\begin{tabular}{|c|c|c|c|}
	\hline
	$A$ & $B$ & $A$ & $B$ \\
	\hline
	$F$ & $F$ & $F$ & $F$ \\
	\hline
	$F$ & $T$ & $F$ & $T$ \\
	\hline
	$T$ & $F$ & $T$ & $F$ \\
	\hline
	$T$ & $T$ & $T$ & $T$ \\
	\hline
\end{tabular}

\end{center}
Thus, we have proven the base case.\\
Inductive hypothesis. Let $\varphi$ be any formula using $A, B$ and at most $n$ $\oplus$s. Then, $\varphi$ has an even number of $T$s in its truth table.\\
Inductive step. Let $\varphi$ be a formula with $n+1$ $\oplus$s. As $\varphi$ is built using only $\oplus$s, we can write $\varphi$ as $\varphi_1 \oplus \varphi_2$ for some formulae $\varphi_1$ and $\varphi_2$ which have at most $n$ $\oplus$s. By induction hypothesis, $\varphi_1$ has $2a$ $T$s in its truth table and $\varphi_2$ has $2b$ $T$s for some integers $a$ and $b.$\\
Let $i$ denote the number of lines in which a $T$ of $\varphi_1$ is paired with an $F$ of $\varphi_2$ and let $j$ denote the number of lines in which an $F$ of $\varphi_1$ is paired with a $T$ of $\varphi_2.$ Then, $i + j$ is the total number of $T$s in the truth table of $\varphi.$\\
Now note that the remaining $(2a - i)$ $T$s of $\varphi_1$ were paired with the remaining $(2b - j)$ $T$s of $\varphi_2.$ Thus, $2a - i = 2b - j.$ Simple algebra gives us that $i + j = 2(a - b + j),$ which is even, as desired.\\
This proves our claim.\\
The claim then completes the proof as $A \wedge B$ has an odd number of $T$ in its truth table and therefore, cannot possibly be written using just $\oplus.$

\hrulefill

5. First we show that consistency implies satisfiability.\\
If $\mathcal{F}$ is consistent, then $\mathcal{F} \not\vdash \perp,$ by definition.\\
As propositional logic is sound and complete, this means that $\mathcal{F} \not\models \perp.$\\
What the above means is that there is an assignment $\alpha$ such that $\alpha \models \mathcal{F}$ and $\alpha \not\models\perp.$ The latter is always true anyway but the existence of an $\alpha$ such that $\alpha \models \mathcal{F}$ shows that $\mathcal{F}$ is satisfiable.\\~\\
Conversely, assume that $\mathcal{F}$ is satisfiable. Then there exists an assignment $\alpha$ such that $\alpha \models \mathcal{F},$ by definition. As $\perp$ is never true, $\alpha \not \models \perp.$ Thus, we have it that $\mathcal{F}\not\models \perp.$ (Since there exists at least one assignment such that $\mathcal{F}$ is true but $\perp$ is not.)\\
As propositional logic is sound and complete, $\mathcal{F} \not \vdash \perp.$ \\
Thus, $\mathcal{F}$ is consistent, by definition.

\hrulefill

6. %(a) Since $\mathcal{F}$ is given to be inconsistent
(b) We are given that $\mathcal{F} \vdash \perp.$ \hfill (Definition of inconsistency.)\\
Also, $\mathcal{F} = \mathcal{F}_G \cup \{G\}.$ Thus, $\quad \mathcal{F}_G, \;G \vdash \perp.$\\
The above is the same as $\mathcal{F}_G \vdash G \to \perp.$ \hfill (Done in class.)\\
As $G \to \perp \equiv \neg G,$ we get that $\mathcal{F}_G \vdash \neg G.$
\end{document}